\documentclass{article}

\usepackage{booktabs}
\usepackage{tabularx}

\title{SE 3XA3: Development Plan\\Title of Project}

\author{Team \#, Team Name
		\\ Student 1 name and macid
		\\ Student 2 name and macid
		\\ Student 3 name and macid
}

\date{}

\input{../Comments}

\begin{document}

\begin{table}[hp]
\caption{Revision History} \label{TblRevisionHistory}
\begin{tabularx}{\textwidth}{llX}
\toprule
\textbf{Date} & \textbf{Developer(s)} & \textbf{Change}\\
\midrule
Date1 & Name(s) & Description of changes\\
Date2 & Name(s) & Description of changes\\
... & ... & ...\\
\bottomrule
\end{tabularx}
\end{table}

\newpage

\maketitle

Put your introductory blurb here.

\section{Team Meeting Plan}

\paragraph{}
Team meetings will occur once per week for about thirty minutes after our first lab of the week. Our first lab is on Wednesdays from 8:30 to 10:20 in ITB , and we will have our meeting at the café in ETB immediately after the lab. This way all of us can easily attend the meeting since we will all have met during the lab. We will have a team chair to keep the meeting on topic, a scribe who will keep record of the meeting with a meeting agenda, and the last person will keep track of how well the implementation and program requirements fit together.\par The meeting agenda will divide the meeting into three equal parts - reviewing the progress for the current week's deliverables, discussing any problems/changes with the implementation and documents, and dividing the work to be done for the next week's deliverables. This agenda is general and topics are subject to change over time.

\section{Team Communication Plan}

\section{Team Member Roles}

\section{Git Workflow Plan}

\paragraph{}
Our team will be developing our program using the Centralized Workflow plan, keeping a central repository to serve as a single point-of-entry for any and all changes made to the project (changes will be committed to the default \textit{master} branch). Since we are all relatively new to using Git, this will help us easily keep track of each other's changes to our project files. We will use tags to keep track of important deliverables and personal milestones for the project, such as when major portions of our program may be implemented to work. Tracking these commits will ensure that we can fallback should we come across issues with our game, and that we can have separate releases if we are able to implement extra features.

\section{Proof of Concept Demonstration Plan}

\section{Technology}

\section{Coding Style}

\section{Project Schedule}

Provide a pointer to your Gantt Chart.

\section{Project Review}

\end{document}