\documentclass{article}

\usepackage{tabularx}
\usepackage{booktabs}

\title{SE 3XA3: Problem Statement\\NAMCAP}

\author{Team 2, VPB Game Studio
		\\ Prajvin Jalan (jalanp)
		\\ Vatsal Shukla (shuklv2)
		\\ Baltej Toor (toorbs)
}

\date{2016-09-23}

\input{../Comments}

\begin{document}

\begin{table}[hp]
\caption{Revision History} \label{TblRevisionHistory}
\begin{tabularx}{\textwidth}{llX}
\toprule
\textbf{Date} & \textbf{Developer(s)} & \textbf{Change}\\
\midrule
2016-09-22 & Baltej Toor & Begin write-up of Problem Statement using the given
template\\
2016-09-23 & Baltej Toor & Add problem statement, reasoning and context content
to document\\
2016-09-23 & Baltej Toor & Add decided Team Name\\
... & ... & ...\\
\bottomrule
\end{tabularx}
\end{table}

\newpage

\maketitle

\paragraph{}
Videogames provide a unique, interactive entertainment experience that takes
players away from their boredom and stress of life. Modern day game developers
look to create an immersive environment in which to capture the user with
detailed graphics and gameplay. Classical arcade games such as Space Invaders,
Asteroids, and Pacman among many others captivated gamers with their simple
design and mechanics. As the redevelopment team, we plan on developing a game
that recaptures the simplicity and playability of class arcade games to combat
daily stress and boredom for our users.

\paragraph{}
Modern videogames attempt to immerse users with complex graphical and technical
features. Unfortunately many of these features come at increasing cost that
serves as a barrier-to-entry for many players. If users want a more simplistic
arcade interaction the costs are even greater. Through the redevelopment of a
classic arcade game, Pacman, we are looking to bring a straightforward,
enjoyable game to the hands of gamers without the need to acquire older
generation arcade machines.

\paragraph{}
The conceptual redevelopment of Pacman, Namcap is not age-restricted or limited
to users with previous knowledge of complex gaming mechanics. Gamers looking to
try Namcap for an arcade-like experience take stake in the project along with
the redevelopment team. Namcap will provide a free playable desktop app
independent of the user's platform allowing for easy accessibility to a wide
user base. By taking the appropriate steps to document and test the project, we
ensure open-source developers can efficiently maintain and improve on the design
in the future.

\end{document}